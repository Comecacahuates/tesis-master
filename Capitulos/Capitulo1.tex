%-----------------------------------
%   INTRODUCCIÓN
%-----------------------------------
\chapter{Introducción}

\label{ch:introduccion}

La evolución de los sistemas de comunicación se ha acelerado importantemente en
las últimas décadas. El primer gran paso en el desarrollo de estos sistemas fue
el uso de ondas de radiofrecuencia para enviar mensajes entre un transmisor y
un receptor. Después, se crearon protocolos y dispositivos más complejos para
formar redes que permitieron el intercambio de información entre más de dos
computadoras. Esto dio origen a la Internet, que hace posible que millones de
dispositivos puedan compartir información desde casi cualquier parte del mundo.

Con el desarrollo de los dispositivos móviles, como los teléfonos inteligentes,
las redes celulares se han convertido en uno de los sistemas de comunicación
más usados en el mundo. Estas redes permiten tener conectividad incluso estando
en movimiento. Sin embargo, la conexión está restringida al área de cobertura de
las antenas. Existen lugares donde no se cuenta con infraestructura de red, por
lo que no hay posibilidad de comunicarse aunque se cuente con un dispositivo
móvil. Debido a estas limitaciones, es natural considerar a las redes móviles
sin infreastructura fija como el siguiente paso en la evolución de los sistemas
de telecomunicaciones.

Las redes móviles \textit{ad hoc}, o MANETs (\textit{mobile ad hoc networks}),
se forman únicamente por dispositivos móviles. Estas redes no cuentan con
infraestructura fija, por lo que los dispositivos deben configurse de manera
autónoma para comunicarse entre sí y formar una red. No obstante, las
características de estas redes implican nuevos retos que se tienen que superar
para proporcionar un servicio confiable. Su principal característica es que no
tienen una topología fija, ya que los nodos tienen la libertad de moverse sin
restricciones. Esto hace que el enrutamiento sea el problema más difícil de
resolver \cite{Li2008}.

Un tipo de MANET que ha despertado gran interés en los últimos años, son las
redes \textit{ad hoc} vehiculares, o VANETs (\textit{vehicular ad hoc
networks}). Cada vehículo cuenta con un dispositivo que le permite comunicarse
con los demás para formar la red. Debido a que los vehículos se mueven a altas
velocidades y cada uno sigue una ruta distinta, los enlaces tienen duraciones
muy cortas. Es por esto que la topología cambia con más frecuencia y el
enrutamiento se vuelve aún más complicado \cite{Meneguette2018}.

Existen muchos tipos de aplicaciones para las VANETs, que se pueden clasificar
en dos principales grupos. Las aplicaciones de seguridad tienen como objetivo
proporcionar a los conductores información útil para prevenir accidentes. Las
aplicaciones de no-seguridad se enfocan en servicios como compartir y descargar
archivos, monitorización del tráfico, mensajería, entre otros
\cite{Meneguette2018}.

El objetivo de este trabajo es diseñar un sistema de comunicación basado en una
VANET que permita a las personas compartir información. Esto sería
particularmente útil en lugares o situaciones en las que no haya disponible una
red de infraestructura fija que permitan a las personas mantenerse comunicarse.
Para esto, se propone un protocolo de enrutamiento para VANETs que cuente con
un mecanismo que le proporcione a cada dispositivo la capacidad de configurarse
autónomamente.

%-----------------------------------
%   FORMULACIÓN DEL PROBLEMA
%-----------------------------------
\section{Formulación del problema}

\label{sec:formulacion_del_problema}

Para que la red pueda proporcionar un buen servicio, se necesita que la mayor
cantidad de paquetes lleguen a su destino en el menor tiempo posible. Los
protocolos de enrutamiento son los principales responsables de que se cumplan
estas condiciones. En las redes cableadas, la topología sufre cambios muy rara
vez, por lo que se pueden determinar las rutas conociendo la topología.
Pero en las MANETs ocurre exactamente lo opuesto, por lo que el enrutamiento es
considerablemente más complicado.

En las MANETs, los enlaces inalámbricos entre los dispositivos pueden tener
duraciones muy cortas como consecuencia del constante movimiento de los
dispositivos. Por esta razón, una vez que se encuentra una ruta hacia algún otro
dispositivo en la red, toda la ruta quedaría inutilizable tan sólo con
desaparecer uno de los enlaces que la forman. Este problema se acentúa aún más
en las VANETs, ya que los vehículos se mueven a velocidades aún más altas.

Se han desarrollado diferentes enfoques para resolver el problema del
enrutamiento en este tipo de redes. Uno en particular que ha mostrado buenos
resultados implica considerar la ubicación de los dispositivos en lugar de la
topología a la hora de buscar una ruta. El remitente indica en cada paquete la
dirección y la ubicación del destinatario, lo que es de mucha ayuda para
determinar hacia dónde se debe transmitir. Además de esto, apoyarse de mapas ha
servido para mejorar la calidad del enrutamiento \cite{Lochert2003}. Sin
embargo, se presenta otro problema: cómo saber la ubicación del destinatario
\cite{Brendha2017}.

Por otro lado, antes de poder comenzar a hacer el enrutamiento, es
indispensable que cada nodo de la red tenga una dirección, ya que es la manera
en la que se identifica ante los demás. Las direcciones permiten saber qué nodo
envía un paquete y a qué nodo va destinado. En las redes de infraestructura
fija, la asignación de direcciones se hace de manera centralizada, generalmente
mediante un servidor DHCP (\textit{Dynamic Host Configuration
Protocol})\footnote{Protocolo de configuración dinámica de \textit{hosts}.}.
Hacerlo de este modo garantiza que ninguna dirección esté asignada a dos nodos
al mismo tiempo. Sin embargo, en una VANET no es tan simple lograr que un solo
nodo administre las direcciones de todos los demás \cite{Korichi2018}.

Dicho lo anterior, para poder desplegar una VANET que proporcione un servicio de
comunicación, lo primero que se necesita es un mecanismo descentralizado que
permita que los dispositivos obtener una dirección única. Una vez que los
dispositivos cuenten con una dirección, antes de enviar un paquete, se necesita
un método que les permita obtener la ubicación del destinatario. Una vez que
tienen la ubicación del destinatario, se necesita que un protocolo de
enrutamiento haga llegar los paquetes al dispositivo correspondiente mediante
saltos entre dispositivos intermedios.

%-----------------------------------
%   PREGUNTAS DE INVESTIGACIÓN
%-----------------------------------

\section{Preguntas de investigación}

\label{sec:preguntas_de_investigacion}

\begin{itemize}
  \item ¿Cómo se puede realizar la configuración de direcciones de los nodos en
  una VANET?
  \item ¿Cómo se pueden determinar las rutas en una VANET con ayuda de un mapa?
\end{itemize}

%----------------------------------------------------------------------------------------
%   OBJETIVOS
%----------------------------------------------------------------------------------------

\section{Objetivos}

\label{subsec:objetivos}

%----------------------------------------------------------------------------------------
%   OBJETIVO GENERAL
%----------------------------------------------------------------------------------------

\subsection{Objetivo general}
\label{subsec:objetivo_general}

Diseñar un protocolo de enrutamiento para VANETs que permita usar la información
de las vialidades para determinar las rutas y que permita a cada vehículo
configurar su propia dirección única.

%----------------------------------------------------------------------------------------
%   OBJETIVOS ESPECÍFICOS
%----------------------------------------------------------------------------------------

\subsection{Objetivos específicos}
\label{subsec:objetivos_especificos}

\begin{itemize}
  \item Diseñar el protocolo de autoconfiguración de direcciones.
  \item Diseñar el protocolo de enrutamiento.
  \item Evaluar el desepeño del protocolo mediante simulaciones.
\end{itemize}

%----------------------------------------------------------------------------------------
%   JUSTIFICACIÓN
%----------------------------------------------------------------------------------------

\section{Justificación}
\label{sec:justificacion}

Debido a las características de las VANETs, la diversidad de aplicaciones que
tienen y los diferentes escenarios en los que se pueden aplicar, el enrutamiento
es un enorme reto que no se ha logrado superar por completo. No existen
protocolos de enrutamuiento bien consolidados para estas redes, como sí los hay
para las redes de infraestructura fija. Por esta razón, este es un tema de
investigación muy activo actualmente.

La mayoría de estos protocolos se enfocan en aplicaciones como seguridad vial,
monitorización del tráfico, entretenimiento, entre otras. El protocolo
propuesto en este trabajo se enfoca en una aplicación diferente, que es
proporcionar un servicio de comunicación alternativo en caso de no tener acceso
a redes de infraestructura fija.

Existe una gran cantidad de protocolos para VANETs que se han propuesto en los
últimos años. Con el desarrollo de nuevos protocolos, eventualmente se
logrará ratificar estándares para diferentes aplicaciones o condiciones.
